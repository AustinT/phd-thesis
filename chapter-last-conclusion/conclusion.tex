\chapter{Conclusions and Future Work}
\label{chapter:conclusion}

This chapter summarises the contributions of this thesis
(\S\ref{sec:conclusion:contributions})
and reviews directions for future work
(\S\ref{sec:conclusion:future-work}).
Finally, in \S\ref{sec:conclusion:thoughts}
I take the opportunity to reflect more broadly
on the state of machine learning for molecule discovery
as a whole.

\section{Summary of Contributions}
\label{sec:conclusion:contributions}

Molecule discovery generally proceeds through a design-make-test-analyse (DMTA) cycle
(introduced in \S\ref{sec:intro:contributions}).
This thesis introduced several probabilistic machine learning algorithms
to automate and improve different stages of this cycle.

Chapter~\ref{chapter:lso} described
\emph{latent space optimisation with weighted retraining},
which uses a Gaussian process model trained on molecule embeddings
from a deep generative model to select molecules with potentially desirable properties,
and refines the embeddings over time by retraining the generative model
on a weighted version of the dataset.
Experimentally, this method outperformed other latent space optimisation methods
which did not use weighted retraining.
However, the method was limited by the fit of the Gaussian process model.

Chapter~\ref{chapter:adkf} tried to compensate for this limitation
by introducing \emph{adaptive deep kernel fitting with implicit function theorem} (ADKF-IFT):
a novel way to fit deep kernel Gaussian process models in the meta-learning settings.
By meta-learning some (but not all) of the model's parameters,
ADKF-IFT was able to achieve a balance between overfitting and underfitting
which was not possible with the established methods for fitting deep kernel GPs,
which were either entirely based on meta-learning or not at all.
Our experiments showed that ADKF-IFT achieved the overall best performance
on the FS-Mol meta-learning benchmark.
A by-product of ADKF-IFT is meta-learned molecule embeddings,
which could be used in other out-of-distribution machine learning tasks.

Chapter~\ref{chapter:trf} introduced \emph{Tanimoto random features}:
a fast approximation for Gaussian processes with the Tanimoto kernel.
Our experiments showed that the random features were indeed able to accurately
approximate realistic matrices of Tanimoto coefficients between molecules.
This method could be especially useful for problems with small \emph{training} datasets
(too small to fit a deep kernel Gaussian process)
but large \emph{test} datasets.
Tanimoto random features could, for example,
allow Bayesian optimisation with approximate Thompson sampling to be applied
to the ultra-large ZINC database,
which contains over $10^9$ molecules.

Finally, Chapter~\ref{chapter:rfb} introduced \emph{retro-fallback}:
an algorithm to search for synthesis plans which explicitly accounts for uncertainty.
Noting that any individual synthesis plan may fail,
the algorithm is explicitly designed to maximise the probability
that at least one synthesis plan it finds will be successful
(a metric called \emph{successful synthesis probability} or SSP).
It operates by using dynamic programming to estimate how expanding
individual molecules in the search graph could improve SSP,
then takes actions greedily.
Experimentally, we showed that retro-fallback
maximised SSP more effectively than the established retro* and MCTS algorithms.
However, these experiments used an unproven model to predict reaction failure,
leaving open the question of how well retro-fallback will perform in practice.

The relationship between these contributions and the DMTA cycle
is illustrated in Figure~\ref{fig:dmta-schematic-conclusion}
(a copy of Figure~\ref{fig:dmta-schematic} from Chapter~\ref{chapter:intro}).
The methods from
chapters~\ref{chapter:lso}, \ref{chapter:adkf}, and \ref{chapter:trf}
are all models to analyse data and predict properties of molecules,
allowing molecules with potentially desirable properties to be selected
in the design stage.
Retro-fallback (chapter~\ref{chapter:rfb}) can help in the design stage
by avoiding molecules which are not easy to synthesize.
Once a suitable molecule is designed, synthesis plans
from retro-fallback can be used to \emph{make} the molecule.
Together, all of these methods could be used in a \emph{Bayesian optimisation} loop
(\S\ref{sec:background:bayesopt}).

\begin{figure}[htb]
    \begin{tikzpicture}[node distance=3cm,>={Stealth[scale=2]}]
    \node (design) [draw, rectangle, minimum width=3cm, minimum height=1cm] {Design};
    \node (make) [draw, rectangle, minimum width=3cm, minimum height=1cm, right=of design] {Make};
    \node (test) [draw, rectangle, minimum width=3cm, minimum height=1cm, below=of make] {Test};
    \node (analyse) [draw, rectangle, minimum width=3cm, minimum height=1cm, below=of design] {Analyse};
    
    \draw [->] (design) -- (make);
    \draw [->] (make) -- (test);
    \draw [->] (test) -- (analyse);
    \draw [->] (analyse) -- (design);
    
    \draw [decorate, decoration={brace, amplitude=10pt, raise=5pt}, yshift=0pt, xshift=0] 
    (analyse.south west) -- (design.north west) node [black, midway, xshift=-2cm, align=left] {
        Chapters~\ref{chapter:lso}, \ref{chapter:adkf}, \ref{chapter:trf}: \\
        Property \\ prediction \\ models \\
        (defined \S\ref{sec:background:models}) \\
    };

    \draw [decorate, decoration={brace, amplitude=10pt, raise=5pt}, rotate=90] 
    (design.north west) -- (make.north east) node [black, midway, xshift=0, yshift=1cm, align=left] {
        Chapter~\ref{chapter:rfb}: retrosynthesis \\
        (defined \S\ref{sec:background:synthesis}) \\
    };
\end{tikzpicture}

    \caption[Repeated schematic of the DMTA cycle]{
        DMTA cycle for molecule discovery
        and how contributions from this thesis fit into it.
        NOTE: this figure is an identical copy of Figure~\ref{fig:dmta-schematic}.
    } 
    \label{fig:dmta-schematic-conclusion}
\end{figure}



\section{Future Work}
\label{sec:conclusion:future-work}

\paragraph{Better evaluation}
There are plenty of ways to modify the algorithms in this thesis
to potentially be more useful for molecule discovery.
However, I think the most important direction for future work is
actually to improve the \emph{evaluation} of existing algorithms.
Most works in this space test molecule optimisation algorithms
using toy datasets or functions, ranging from the simple logP/QED functions
to more complicated docking scores \citep{ortegon2021dockstring}.
Critically, it is unclear how well these toy functions
represent the actual molecule discovery tasks that the algorithms are supposed to solve.
My personal opinion is that they are \emph{not} very representative,
chiefly because toy functions can be unimodal, smooth (with respect to changes in molecular structure),
and have simple structure-property relationships which can be exploited by algorithms.
In order to make progress, it is important to know the shortcomings of existing algorithms
on \emph{real} tasks, not just toy ones.
In my opinion, most directions for future work are downstream of this one.

\paragraph{Implement and test ``simple'' Bayesian optimisation baselines}
The vast majority of Bayesian optimisation algorithms for molecule discovery
which get published in machine learning conferences are complicated. They often use
custom kernels and acquisition functions, and essentially describe their algorithm
as a complete package rather than a combination of modular components which can be
freely interchanged. This seems to give the impression that Bayesian optimisation
algorithms must be \emph{complicated} to work well.
I think this is not true--- I think simpler solutions have just not been tried,
or if they have been tried then they are not published in machine learning conferences
(probably because there would not be much technical novelty).
I think the community would really benefit from a systematic study of existing
simple methods (e.g.\@ Bayesian optimisation with Tanimoto kernel Gaussian processes).

\paragraph{Better interface between algorithms and real world}
The algorithms in this thesis are unified by their potential inclusion
in Bayesian optimisation algorithms for molecule optimisation.
In reality however, for these algorithms to be used there must be an interface
between the algorithm and the data collection procedure.
This could take two forms, both of which pose additional challenges:
\begin{enumerate}
    \item \textbf{Human-in-the-loop:}
        a human would be doing the experiments and feeding the results back into the algorithm.
        Few chemists are accustomed to this workflow however,
        and if they do not \emph{trust} or \emph{understand} the algorithm
        they will be hesitant to use it (or may not fully follow its suggestions).
        To remedy this, a lot more thought probably needs to be put into the interface
        between the algorithm and the human.
        This could be in the form of interpretability or explainability (which could help build trust),
        or by designing the algorithm to explicitly tolerate a human editing its outputs.
    \item \textbf{Lab robots:}
        also called ``self-driving laboratories,''
        lab robots can potentially do experiments at large scales and
        without human supervision
        \citep{seifrid2022autonomous,koscher2023autonomous,tom2024sdl}.
        This is a more natural fit for the algorithms in this thesis,
        but also has its own challenges.
        For example, lab robots may not be able to correctly intervene if something goes wrong,
        such as stopping a chemical reaction which is getting too hot or noticing contamination
        of a sample.
        To deploy the algorithms in this thesis in a lab robot setting,
        more work studying the robustness of these algorithms is probably necessary.
\end{enumerate}

\paragraph{Applications to proteins and materials design}
Designing proteins and materials is in many ways similar to designing molecules.
These problems all have an enormous (and partially discrete) search space,
expensive evaluations, and domain knowledge.
This makes Bayesian optimisation a natural fit for these problems.
Despite this, I think these subfields do not have as much interaction
or knowledge exchange as they should.
I hope this can change in the near future.

\paragraph{Future of Gaussian processes in molecule discovery}
The preceding chapters all utilized GPs in one way or another.
In the age of neural networks and foundation models, it is fair to ask
whether GPs will still be important for molecule discovery in the future.
My answer is \emph{yes}, for the following reasons:
\begin{enumerate}
    \item GPs naturally provide uncertainty estimates at roughly the same cost as the posterior mean
        (in fact the kernel matrix inverse can be reused).
        This is useful for decision-making, and natural scientists highly appreciate uncertainty estimates.
        In contrast, neural networks are known to easily overfit (especially to small datasets).
    \item The inductive bias of GPs can be controlled and interpreted via its kernel function.
        This is also appreciated by natural scientists, who value interpretability and predicable behaviour from models.
        In contrast, neural networks are notoriously inscrutable.
    \item A GP posterior conditioned on $N$ data points can be efficiently updated to incorporate an additional data point
        with only $\mathcal O(N^2)$ cost (compared to $\mathcal O(N^3)$ cost for complete re-fitting).
        This is useful when data is gathered in small batches (e.g.\@ when performing molecule optimisation).
        In contrast, there is no easy way to update a neural network to incorporate an extra data point;
        except in special cases re-fitting is required.
\end{enumerate}
Of course, GPs have disadvantages, chiefly their scaling with dataset size.
However, since many molecule discovery tasks use small datasets,
this disadvantage will not be as pronounced as in other domains.
Therefore, GPs as a whole stand out as a useful tool for molecular discovery,
and deserve future research attention.

One topic worth highlighting for future research is the use of GPs
with non-Gaussian measurement noise.
In this case, the GP posterior can no longer be computed analytically,
requiring approximate inference methods.
Although many such approximations have been proposed in the literature \citep{hensman2013gaussian},
they do not appear to be widely used in practice,
meaning it is unclear how well they would perform on real molecule discovery tasks (which often feature heavy-tailed noise).
Evaluating and improving such approximation methods in the context of molecules
could be a fruitful direction for future research.

\section{Closing thoughts}
\label{sec:conclusion:thoughts}

In fields like computer vision and natural language processing,
deep learning has unquestionably \emph{revolutionized} the field---
so much so that algorithms from the pre-deep learning era are now considered obsolete.
In my opinion, the same cannot be said for molecule discovery.
There are several reasons for this.

First, molecule discovery is a diverse collection of problems
with small, fragmented, and noisy datasets.
This makes classical machine learning methods more competitive.
In fact, much of my PhD has been a humbling experience of how difficult it can be to improve upon very
basic but well-established methods such as random forests or genetic algorithms.
I think many people in machine learning for molecule discovery are not aware of this,
mainly because they do not put much effort into tuning and debugging these basic methods.

Second, algorithms for molecule discovery must be deployed in a more complex setting
compared to most algorithms in computer vision and natural language processing.
In those fields, algorithms are typically deployed
to do tasks which a single human could do: for example,
suggesting the next word in a sentence or identifying objects in an image.
Molecule discovery is different:
it is done by \emph{teams} of humans who interact with each other
and use a wide range of lab equipment.
The usefulness of an algorithm is therefore not simply its prediction accuracy,
but the overall efficacy of the end-to-end system which includes the algorithm.
This is difficult to define and measure.

Therefore, as much as I would like to say that the algorithms in this thesis are the new
``state-of-the-art'' methods for molecule discovery,
the experiments in this thesis are not sufficient to make that claim
(and I suspect that no single thesis could be).
Instead, I hope that the algorithms in this thesis can be used
alongside existing algorithms as \emph{components} for future molecule discovery pipelines.
Because of this, one could fairly argue that title of this thesis,
``Probabilistic machine learning algorithms for molecule discovery,''
is too grandiose.
Nonetheless, I am proud of the work I have done,
and am confident that I have pushed the frontiers of what is possible,
even if only by a small amount.
