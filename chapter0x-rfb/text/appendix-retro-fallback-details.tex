\section{Details of \texorpdfstring{$\rs(\cdot)$, $\psi(\cdot)$ and $\rho(\cdot)$}{s, ψ and ρ}}
\label{appendix:rfbd:s,psi,rho}

\subsection{
    Cost-minimisation interpretation
}
\label{appendix:s,psi,rho min cost}

The functions $\rs(\cdot)$, $\psi(\cdot)$, and $\rho(\cdot)$
have a cost minimisation interpretation
when transformed by
mapping $f(\cdot)\mapsto-\log{f(\cdot)}$
(using the convention that $\log{0}=-\infty$).
To see, this, we write the transformed equations explicitly.
To simplify notation, we write:
\begin{align}
    \dot b (m) &= -\log{b(m)} \\
    \dot f (r) &= -\log{f(r)} \\
    \dot h (m) &= -\log{h(m)}  \\
    \dot \rs (\cdot) &= -\log{\rs(\cdot)}  \\
    \dot \psi (\cdot) &= -\log{\psi(\cdot)}  \\
    \dot \rho (\cdot) &= -\log{\rho(\cdot)} \ . 
\end{align}

For $\rs(\cdot)$ (equations~\ref{eqn:molecule success}--\ref{eqn:reaction success}),
we have
\begin{align}
    \dot\rs(m;\gG',f,b) &= 
        \min\left[
            \dot b(m),
            \min_{r\in\children_{\gG'}(m)}\dot\rs(r;\gG',f,b)
        \right]\ , \\
    \dot\rs(r;\gG',f,b) &=
        \dot f(r)
        + \sum_{m\in\children_{\gG'}(r)}
            \dot\rs(m;\gG',f,b)
    \ . 
\end{align}
Since $\rs(\cdot)\in\{0, 1\}$,
this corresponds to a pseudo-cost of $0$ if a molecule is synthesizable
and a cost of $+\infty$ otherwise.

For $\psi(\cdot)$ (equations~\ref{eqn:psi molecule}--\ref{eqn:psi reaction})
and $\rho(\cdot)$ (equations~\ref{eqn:rho molecule}--\ref{eqn:rho reaction})
we have:
\begin{align}
    \dot\psi(m;\gG',f,b,h) &= \begin{cases}
        \min\left[
            \dot b(m),
            \dot h(m)
        \right] & m\in\frontier(\gG') \\
        \min\left[
            \dot b(m),
            \min_{r\in\children_{\gG'}(m)}
                \dot\psi(r;\gG',f,b,h)
        \right] & m\notin\frontier(\gG') \\
    \end{cases} \ , \\
    \dot\psi(r;\gG',f,b,h) &= \dot f(r) + \sum_{m\in\children_{\gG'}(r)} \dot\psi(m;\gG',f,b,h)\ . 
\end{align}
\begin{align}
    \dot\rho(m;\gG',f,b,h) &= \begin{cases}
        \dot\psi(m;\gG',f,b,h) & m\text{ is target molecule }\targetmol \\
        \min_{r\in\parents_{\gG'}(m)}\dot\rho(r;\gG',f,b,h) & \text{all other }m \\
    \end{cases}\ , \\
    \dot\rho(r;\gG',f,b,h) &= \begin{cases}
        +\infty & \dot\psi(r;\gG',f,b,h) = +\infty \\
        \dot\rho(m';\gG',f,b,h) \\ \qquad- \dot\psi(m';\gG',f,b,h) \\ \qquad + \dot\psi(r;\gG',f,b,h) & \dot\psi(r;\gG',f,b,h) < \infty, m'\in\parents_{\gG'}(r)
    \end{cases}\ .
\end{align}
Since $\psi(\cdot)$ and $\rho(\cdot)$ have ranges $[0,1]$,
$\dot\psi(\cdot)$ and $\dot\rho(\cdot)$ have ranges $[0,+\infty]$
(i.e.\@ explicitly including infinity).


\subsection{
    Clarification of
    \texorpdfstring{$rs$, $\psi$, and $\rho$}{s, ψ and ρ}
    for cyclic graphs
}

If $\gG'$ is an acyclic graph,
the recursive definitions
of $\rs(\cdot)$, $\psi(\cdot)$, and $\rho(\cdot)$
will always have a unique solution obtained by direct recursion
(from children to parents for $\rs(\cdot)$ and $\psi(\cdot)$,
and from parents to children for $\rho(\cdot)$).
However, for cyclic graphs, such recursion is not possible
and therefore these equations do not (necessarily) uniquely define
the functions  $\rs(\cdot)$, $\psi(\cdot)$, and $\rho(\cdot)$.

The minimum cost interpretation from section~\ref{appendix:s,psi,rho min cost}
provides a resolution.
Minimum cost solutions in AND/OR graphs
with cycles have been extensively
studied
\citep{hvalica1996andorsearch,jimenez2000efficient}
and do indeed exist.
\citet[Lemma 1]{hvalica1996andorsearch} in particular suggests
that the presence of multiple solutions will happen only
because of the presence of zero-cost cycles
(which could be caused by a cycle of feasible reactions).
To produce a unique solution,
we re-define all costs to 0 to be a constant $\epsilon>0$
(for which there will be a unique solution for $\dot\rs$, $\dot\psi$, and $\dot\rho$),
and take the limit as $\epsilon\to0$ to produce a unique solution which includes 0 costs.

\subsection{
    Computing
    \texorpdfstring{$\rs$, $\psi$, and $\rho$}{s, ψ and ρ}
}

The best method of computation will depend on $\gG'$.
If $\gG'$ is acyclic (for example, as will always be the case for tree-structured search graphs)
then direct recursion is clearly the best option.
This approach also has the advantage that the computation can stop early if
the value of a node does not change.

In graphs with cycles, any number of algorithms to find minimum
\emph{costs} in AND/OR graphs
can be applied to the minimum-cost formulations
of $\rs(\cdot)$, $\psi(\cdot)$, and $\rho(\cdot)$
from section~\ref{appendix:s,psi,rho min cost}
\citep{chakrabarti1994algorithms,hvalica1996andorsearch,jimenez2000efficient}.

\subsection{Analytic solutions for \texorpdfstring{$\psi(\cdot)$ and $\rho(\cdot)$}{ψ and ρ}}

The analytic solutions for $\psi(\cdot)$ and $\rho(\cdot)$
follow straightforwardly from their cost-minimisation interpretation (\ref{appendix:s,psi,rho min cost}).
$\rs$ and $\rho$ are equivalent to $c^*$ with a modified definition of cost,
(equations~\ref{eqn:retro* min cost mol}--\ref{eqn:retro* min cost rxn}),
while $\psi$ is equivalent to $\hat c^*$
(equations~\ref{eqn:retro* min cost mol with heuristic}--\ref{eqn:retro* min cost rxn with heuristic}).
