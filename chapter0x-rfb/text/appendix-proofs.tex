\section{Proofs and Theoretical Results}\label{appendix:proofs}

This appendix contains proofs of theoretical results from the chapter.

\subsection{Proof of Theorem~\ref{thm:success prob is NP hard}}\label{appendix:proofs:succ-prob-np-hard}

Theorem~\ref{thm:success prob is NP hard} is a corollary of the following theorem,
which we prove below.

\begin{theorem}\label{thm:success prob non-zero NP hard}
    Unless $P=NP$, there does not exist an algorithm to determine
    {\color{blue}whether $\mathrm{SSP}(\enumplans_{\targetmol}(\gG') ; \xi_f, \xi_b) > 0$}
    for arbitrary $\xi_f,\xi_b$ whose time complexity grows polynomially with the number of nodes
    in $\gG'$.
\end{theorem}
Note that Theorem~\ref{thm:success prob non-zero NP hard}
is similar to but distinct from Theorem~\ref{thm:success prob is NP hard}.
The difference ({\color{blue}highlighted})
is that Theorem~\ref{thm:success prob is NP hard}
is about the difficulty of computing SSP,
while Theorem~\ref{thm:success prob non-zero NP hard}
is only about the difficulty of determining whether or not SSP is 0.
We now state a proof of Theorem~\ref{thm:success prob non-zero NP hard}:
\begin{proof}
We will show a reduction from the Boolean 3-Satisfiability Problem (3-SAT) to the problem of determining whether SSP is non-zero. As 3-SAT is known to be NP-hard \citep{Karp1972}, this will imply the latter is also NP-hard, completing the proof.

To construct the reduction, assume an instance $I$ of 3-SAT with $n$ variables $x_1$, \ldots, $x_n$, and $m$ clauses $c_1$, \ldots, $c_m$, each $c_j$ consisting of three literals (where a literal is either a variable or its negation).
We will construct an AND-OR graph $\gG(I)$ with size $\mathcal O(n + m)$, alongside with distributions $\xi_f(I)$~and $\xi_b(I)$,
such that the SSP in the constructed instance is non-zero if and only if $I$ is satisfiable.

In our construction we first set $\xi_f \equiv 1$, i.e.\@ assume all reactions described below are always feasible.

We then construct a set $P$ of $2n$ potentially buyable molecules,
corresponding to variables $x_i$ as well as their negations $\neg x_i$;
to simplify notation, we will refer to these molecules as $x_i$ or $\neg x_i$.
We then set $\xi_b(I)$ to a uniform distribution over all subsets $S \subseteq P$ such that $|S \cap \{x_i, \neg x_i\}| = 1$
for all $i$;
in other words, either $x_i$ or $\neg x_i$ can be bought, but never both at the same time.
Note that with this construction it is easy to support all necessary operations on $\xi_b$, such as (conditional) sampling or computing marginals.

It remains to translate $I$ to $\gG(I)$ in a way that encodes the clauses $c_j$. 
We start by creating a root OR-node $r$, with a single AND-node child $r'$.
Under $r'$ we build $m$ OR-node children, corresponding to clauses $c_j$; again, we refer to these nodes as $c_j$ for simplicity.
Finally, for each $c_j$, we attach $3$ children, corresponding to the literals in $c_j$.
Intuitively these $3$ children would map to three molecules from the potentially buyable set $P$, but formally the children of $c_j$ should be AND-nodes (while $P$ contains molecules, i.e.\@ OR-nodes); however, this can be resolved by adding dummy single-reactant reaction nodes.

To see that the reduction is valid, first note that $r$ is synthesizable only if all $c_j$ are, which reflects the fact that $I$ is a binary AND of clauses $c_j$. Moreover, each $c_j$ is synthesizable if at least one of its $3$ children is, which translates to at least one of the literals being satisfied. Our construction of $\xi_b$ allows any setting of variables $x_i$ as long as it's consistent with negations $\neg x_i$. Taken together, this means the SSP for $\gG(I)$ is non-zero if and only if there exists an assignment of variables $x_i$ that satisfies $I$, and thus the reduction is sound.
\end{proof}

\begin{corollary}
If a polynomial time algorithm did exist to compute the exact value of
$\mathrm{SSP}(\enumplans_{\targetmol}(\gG');\gG',\xi_f,\xi_b)$,
this algorithm would clearly also determine whether
$$\mathrm{SSP}(\enumplans_{\targetmol}(\gG');\gG',\xi_f,\xi_b)>0$$
in polynomial time,
violating Theorem~\ref{thm:success prob non-zero NP hard}.
This proves Theorem~\ref{thm:success prob is NP hard}.
\end{corollary}

\subsection{
    Computing
    \texorpdfstring{$\rs(\cdot)$, $\psi(\cdot)$ and $\rho(\cdot)$}{s, ψ and ρ}
    polynomial time
}
\label{appendix:proofs:polynomial-dp-updates}

Here, we state provide proofs that these quantities can be computed in polynomial time.
These proofs require the following assumption:
\begin{assumption}[Bound on children size]
\label{assumption:bound on graph edges}
    Assume 
    \begin{equation*}
      |\children_\gG(m)|\leq K,\quad
      |\children_\gG(m)|\leq L, \quad
      |\parents_\gG(m)|\leq J
    \end{equation*}
    for all molecules $m$ and reactions $r$,
    where $J,K,L\in\mathbb N$ are fixed constants.
\end{assumption}
This assumption essentially ensures that even as the retrosynthesis graph grows,
the number of neighbours for each node does not grow.

Now we state our main results.
\begin{lemma}[$\rs$ and $\psi$ in polynomial time]
\label{thm:s and psi in poly time}
    There exists an algorithm to calculate \\
    $\rs(n;\gG',f,b)$ and $\psi(n;\gG',f,b,h)$
    for every node $n\in\gG'$ whose time
    complexity is polynomial in the number of nodes in $\gG'$.
\end{lemma}
\begin{proof}
    Section~\ref{appendix:s,psi,rho min cost}
    showed how $-\log{\rs(\cdot)}$
    and $-\log{\psi(\cdot)}$
    correspond exactly cost equations for minimum-cost synthesis plans,
    as studied for the AO* algorithm.
    \citep{chakrabarti1994algorithms} provides two algorithms,
    \texttt{Iterative\_revise}
    and \texttt{REV*}
    whose runtime is both worst-case polynomial in the number of nodes
    (assuming a number of edges which does not grow more than linearly
    with the number of nodes,
    which assumption~\ref{assumption:bound on graph edges} guarantees).
\end{proof}

The following lemma shows that,
with a bit of algebraic manipulation,
the same strategy can be applied to $\rho$.
\begin{lemma}
    There exists an algorithm to calculate
    $\rho(n;\gG',f,b)$
    for every node $n\in\gG'$ whose time
    complexity is polynomial in the number of nodes in $\gG'$.
\end{lemma}
\begin{proof}
    To do this, first compute $\psi(\cdot)$
    for every node
    (which lemma~\ref{thm:s and psi in poly time} states can be done in polynomial time).
    Next, define the function
    $\eta(\cdot)=\log{\frac{\psi(\targetmol;\gG',f,b,h)}{\rho(\cdot;\gG',f,b,h)}}$.
    This is essentially a constant transformation of $\rho(\cdot)$
    at each point (since $\psi(\targetmol;\gG',f,b,h)$ is just a constant).
    Therefore $\eta$ has the analytical solution:
    \begin{align}
        \eta(m;\gG',f,b,h) &= \begin{cases}
        0 & m=\targetmol \\ 
        \min_{r\in\parents_{\gG'}(m)}\eta(r;\gG',f,b,h) & \text{all other }m \\
        \end{cases}  \\
        \eta(r;\gG',f,b,h) &= \begin{cases}
            \infty & \psi(r;\gG',f,b,h) = 0 \\
            \eta(m';\gG',f,b,h) + \log{\frac{\psi(m';\gG',f,b,h)}{\psi(r;\gG',f,b,h)}} & \text{otherwise}, m'\in\parents_{\gG'}(r)\ .
        \end{cases}
    \end{align}

    Now, define the graph $\tilde\gG'$ by flipping the direction of all edges
    in $\gG'$ (so all parents become children, and vice versa).
    $\eta$ can be re-written as:
    \begin{align}
        \eta(m;\tilde\gG',f,b,h) &= \begin{cases}
        0 & m=\targetmol \\ 
        \min_{r\in\children_{\tilde\gG'}(m)}\eta(r;\tilde\gG',f,b,h) & \text{all other }m \\
        \end{cases}  \\
        \eta(r;\tilde\gG',f,b,h) &= \begin{cases}
            \infty & \psi(r;\tilde\gG',f,b,h) = 0 \\
            \sum_{m'\in\children_{\tilde\gG'}(r)}
                \eta(m';\tilde\gG',f,b,h) + \log{\frac{\psi(m';\gG',f,b,h)}{\psi(r;\gG',f,b,h)}} & \psi(r;\gG',f,b,h) < \infty \ , \label{eqn:eta tilde for reactions}
        \end{cases}
    \end{align}
    where the sum in equation~\ref{eqn:eta tilde for reactions} was introduced because reactions in $\gG'$
    have only one parent.
    These equations correspond precisely to minimum cost equations from AO*,
    allowing algorithms from \citet{chakrabarti1994algorithms} to be applied.
    $\rho$ can then be straightforwardly recovered from $\eta$.
\end{proof}

