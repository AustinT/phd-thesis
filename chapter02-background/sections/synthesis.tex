\section{Retrosynthetic planning (retrosynthesis)}
\label{sec:background:synthesis}

Performing a physical measurement on a molecule requires either purchasing it or producing it.
Given that even the largest databases of immediately purchasable molecules (such as Enamine) contain only $10^9$ compounds,
accessing most of chemical space requires producing molecules oneself.
This can be achieved by purchasing precursor molecules and performing one or more chemical reactions,
a process called \emph{synthesis}.

Not all molecules are easy to synthesize however, and previous work has shown that many molecules generated
by machine learning models are difficult to synthesize
\citep{gao2020synthesizability}.
If the space of synthesizable molecules is unknown,
this is equivalent to having an incomplete knowledge of
the space of accessible molecules $\amolspace$,
causing issues for molecular optimisation (\S\ref{sec:background:molopt}).
This motivates the importance of explicitly considering synthesis.

There are a few techniques to do this.
One is to produce some kind of ``synthesizability''
score for a molecule, which estimates how easy it is to synthesize.
Some models of this type include synthetic accessibility (SA) scores
\citep{ertl2009estimation},
SC score \citep{coley2018scscore},
and RA score \citep{thakkar2021retrosynthetic}.
Such scores are typically quick to compute,
but are not always accurate.\footnote{
    A fundamental problem with synthesizability scores
    is that ``synthesizability'' is not well-defined.
    Whether a molecule is ``synthesizable'' or not depends on
    ones resources, expertise, and time.
    A large pharmaceutical company will likely be able to synthesize
    more molecules than a lone graduate student.
}
Another technique is to generate synthesis plans directly,
for example using generative models
\citep{bradshaw2019model,bradshaw2020barking,gottipati2020learning,gao2021amortized}.
These models have the advantage of producing an explicit synthesis plan,
but the disadvantage of not having direct control over what molecule the synthesis plan produces!

Because of this, there is arguably no perfect substitute for explicit retrosynthetic planning:
creating a synthesis plan for a pre-specified target molecule by working backwards.
Retrosynthesis has a long history of automation in chemistry
\citep{vleduts1963concerning,corey1969computer}.
The rest of this section will discuss retrosynthetic planning.

\subsection{Problem definition}

Let $\rxnspace$ denote the space of single-product chemical reactions
which transform a set of \emph{reactant} molecules $R\subseteq\molspace$
into a
\emph{product} molecule in $\molspace$.
A large amount of work has been on training machine learning models
to predict reactant sets
which will produce a given product molecule
\citep{segler2017neural,dai2019retrosynthesis,%
seidl2021modern,chen2021deep,sacha2021molecule,irwin2022chemformer,%
zhong2022root,liu2023fusionretro}.
These models are trained on large datasets of known reactions.
The exact nature of these models is not important for the work presented in this thesis,
and therefore we will represent these models abstractly as
a \emph{backward reaction model} function $B:\gM\mapsto 2^\gR$.

The reactions predicted by a backward reaction model can be used to define
a \emph{reaction graph} $\gG$.
There are generally two kinds of reaction graphs.
The first type is called and AND/OR graph.
It is a directed graph which contains two types of nodes:
molecule nodes (also called OR nodes)
and reaction nodes (also called AND nodes).
Molecule nodes contain outgoing edges towards reactions which produce them,
and reaction nodes contain outgoing edges towards their reactant molecules.
Note that direction of the edges (from products towards reactants)
is the \emph{opposite} of how the reaction would actually be performed
(from reactants to products).
There are two subtypes of AND/OR graphs:
graphs which contain only one node for each molecule and each reaction
(and therefore may contain cycles if reversible reactions are present)
and graphs wherein these cycles are ``unrolled''.
We refer to the former type of search graph as simply a ``graph,''
while the latter type is called a ``tree.''
Examples of an AND/OR graph
and its corresponding
AND/OR tree are shown in Figure~\ref{fig:graphs-vs-trees-schematic}.

\begin{figure}[ht]
    \centering
    \begin{minipage}{.48\textwidth}
    \centering
    
    \begin{tikzpicture}[node distance=0.5cm]
          \node[ornode] (target) at (0, 0) {$\targetmol$};

          \node[andnode, below left = of target] (r1) {$r_1$};
          \draw[->] (target) -- (r1);
          \node[ornode, below = of r1] (ma) {$m_a$};
          \draw[->] (r1) -- (ma);
          
          \node[andnode, below = of ma] (r2) {$r_2$};
          \draw[->] (ma) -- (r2);
          \node[ornode, below = of r2] (mb) {$m_b$};
          \draw[->] (r2) -- (mb);

          \node[andnode, below right = of target] (r3) {$r_3$};
          \draw[->] (target) -- (r3);
          \node[ornode, below right = of r3] (mc) {$m_c$};
          \draw[->] (r3) -- (mc);

          \node[andnode, above = of mc] (r4) {$r_4$};
          \draw[->] (mc) -- (r4);
          \draw[->] (r4) -- (target);
    \end{tikzpicture}

    \vfill
    a) Graph version
\end{minipage}
\begin{minipage}{.48\textwidth}
    \centering
    \begin{tikzpicture}[node distance=0.5cm]
          \node[ornode] (target) at (0, 0) {$\targetmol$};

          \node[andnode, below left = of target] (r1) {$r_1$};
          \draw[->] (target) -- (r1);
          \node[ornode, below = of r1] (ma) {$m_a$};
          \draw[->] (r1) -- (ma);
          
          \node[andnode, below = of ma] (r2) {$r_2$};
          \draw[->] (ma) -- (r2);
          \node[ornode, below = of r2] (mb) {$m_b$};
          \draw[->] (r2) -- (mb);

          \node[andnode, below right = of target] (r3) {$r_3$};
          \draw[->] (target) -- (r3);
          \node[ornode, below = of r3] (mc) {$m_c$};
          \draw[->] (r3) -- (mc);

          \node[andnode, below = of mc] (r4) {$r_4$};
          \draw[->] (mc) -- (r4);
          \node[ornode, below = of r4] (target2) {$\targetmol$};
          \draw[->] (r4) -- (target2);
          \node[andnode, below = of target2] (r3-2) {$r_3$};
          \draw[->] (target2) -- (r3-2);
          \node[labelnode, below = of r3-2] (dots) {$\vdots$};
          \draw[->] (r3-2) -- (dots);
    \end{tikzpicture}

    b) Tree version

\end{minipage}

    \caption[Example of an AND/OR graph and tree.]{
        AND/OR graph and tree representation of the reaction set
        $\targetmol\Rightarrow m_a$ ($r_1$),
        $m_a\Rightarrow m_b$ ($r_2$),
        $\targetmol\Rightarrow m_c$ ($r_3$),
        and
        $m_c\Rightarrow \targetmol$ ($r_4$).
    }
    \label{fig:graphs-vs-trees-schematic}
\end{figure}

A second kind of reaction graph is an OR tree.
This is a directed graph where each node corresponds to a \emph{set}
of molecules,
and outgoing edges correspond to reactions which produce the set of molecules
from another set of molecules (not necessarily using all of the molecules as reactants though).
In this type of graph, cycles are always unrolled so that the graph is always a tree.
An example of an OR tree is shown in Figure~\ref{fig:or-graph-schematic}.

\begin{figure}[ht]
    \centering
    \begin{tikzpicture}[node distance=0.5cm]
      \node[ornode] (target) at (0, 0) {$\{\targetmol\}$};

      \node[ornode, below left = of target] (ma) {$\{m_a\}$};
      \draw[->] (target) -- (ma);
      
      \node[ornode, below = of ma] (mb) {$\{m_b\}$};
      \draw[->] (ma) -- (mb);

      \node[ornode, below right = of target] (mc) {$\{m_c\}$};
      \draw[->] (target) -- (mc);
      
      \node[ornode, below = of mc] (target2) {$\{\targetmol\}$};
      \draw[->] (mc) -- (target2);
      \node[ornode, below = of target2] (mc2) {$\{m_c\}$};
      \draw[->] (target2) -- (mc2);
      \node[labelnode, below = of mc2] (dots) {$\vdots$};
      \draw[->] (mc2) -- (dots);
\end{tikzpicture}

    \caption{OR tree for same set of reactions as Figure~\ref{fig:graphs-vs-trees-schematic}.}
    \label{fig:or-graph-schematic}
\end{figure}

One could imagine forming a ``complete'' reaction graph $\gG$ by starting with a
\emph{target molecule} $\targetmol$ and iteratively ``growing''
a graph by using a backward reaction model $B$ to add reactions
to all leaf molecules until no molecules remain.
This ``complete'' graph $\gG$ would contain all possible synthesis plans,
and therefore be a natural object for retrosynthetic planning.
Unfortunately, the size of this graph will typically grow exponentially
with the depth, and therefore be too large to compute in practice
(analogous to the game trees of chess and Go).
Instead, retrosynthetic planning will be performed
on an \emph{explicit search} $\gG'\subseteq\gG$.
In contrast, the ``complete'' graph $\gG$ will be called the \emph{implicit search graph}.

A \emph{synthesis plan} for a molecule $m$ is a sequence of chemical reactions
which produces $m$ as the final product.
Synthesis plans usually form trees $T\subseteq\gG$
(more generally directed acyclic subgraphs),
wherein each molecule is produced by at most one reaction.
The set of all synthesis plans in $\gG$ which produce a molecule $m$
is denoted $\enumplans_m(\gG)$.
Figure~\ref{fig:retrosynthesis-background-schematic} provides an example
of an explicit search graph (a) and all synthesis plans for $\targetmol$
which are contained within it (b).
Not all synthesis plans are equally useful however.
Most importantly,
for a synthesis plan to actually be executed by a chemist
the starting molecules must all be bought.
Typically this is formalized as requiring all starting molecules
to be contained in an \emph{inventory} $\gI\subseteq\molspace$.
It is also desirable for synthesis plans to have low cost,
fewer steps, and reactions which are easier to perform.

\begin{figure}[ht]
    \centering
    \begin{minipage}{.48\textwidth}
    \centering
    
    \begin{tikzpicture}[node distance=0.5cm]
          \node[ornode] (target) at (0, 0) {$\targetmol$};

          \node[andnode, below left = of target] (r1) {$r_1$};%
          \draw[->] (target) -- (r1);
          \node[ornode, below left = of r1] (ma) {$m_a$};
          \draw[->] (r1) -- (ma);
          \node[ornode, below right = of r1] (mb) {$m_b$};
          \draw[->] (r1) -- (mb);
          
          \node[andnode, below right = of target] (r2) {$r_2$};%
          \draw[->] (target) -- (r2);
          \draw[->] (r2) -- (mb);
          \node[ornode, below = of r2] (mc) {$m_c$};
          \draw[->] (r2) -- (mc);
          \node[ornode, below right = of r2] (md) {$m_d$};
          \draw[->] (r2) -- (md);
          
          \node[andnode, below = of ma] (r3) {$r_3$};%
          \draw[->] (ma) -- (r3);
          \node[ornode, below = of r3] (me) {$m_e$};
          \draw[->] (r3) -- (me);

          \node[andnode, below left = of mc] (rc1) {$r_4$};
          \draw[->] (mc) -- (rc1);
          \node[andnode, below right = of mc] (rc2) {$r_5$};
          \draw[->] (mc) -- (rc2);
          \node[ornode, below = of rc1] (mf) {$m_f$};
          \draw[->] (rc1) -- (mf);
          \node[ornode, below = of rc2] (mg) {$m_g$};
          \draw[->] (rc2) -- (mg);
          \node[draw, dashed, fit=(rc1) (rc2) (mf) (mg), inner sep=6pt, label=below:{expansion of $m_c$}] (box) {};
    \end{tikzpicture}

    \vfill
    a) $\gG'$
\end{minipage}
\begin{minipage}{.48\textwidth}
    \centering

    \begin{tikzpicture}[node distance=0.5cm]
        \node (T1) at (-1,0) {$T_1:$};
        \node[ornode] (target) at (0, 0) {$\targetmol$};
        
        \node (T2) at (-1,-1) {$T_2:$};
        \node[ornode] (target2) at (0, -1) {$\targetmol$};
        \node[andnode, right = of target2] (r1) {$r_1$};
        \draw[->] (target2) -- (r1);
        \node[ornode, above right = of r1] (ma) {$m_a$};
        \draw[->] (r1) -- (ma);
        \node[ornode, right = of r1] (mb) {$m_b$};
        \draw[->] (r1) -- (mb);
        
        \node (T3) at (-1,-2.2) {$T_3:$};
        \node[ornode] (target3) at (0, -2.2) {$\targetmol$};
        \node[andnode, right = of target3] (r1b) {$r_1$};
        \draw[->] (target3) -- (r1b);
        \node[ornode, right = of r1b] (mb2) {$m_b$};
        \draw[->] (r1b) -- (mb2);
        \node[ornode, above right = of mb2, xshift=-.2cm, yshift=-.2cm] (ma2) {$m_a$};
        \draw[->] (r1b) |- (ma2);
        \node[andnode, right = of ma2, xshift=-0.2cm] (r3) {$r_3$};
        \draw[->] (ma2) -- (r3);
        \node[ornode, right = of r3, xshift=-0.2cm] (me) {$m_e$};
        \draw[->] (r3) -- (me);

        \node (T4) at (-1,-3.5) {$T_4:$};
        \node[ornode] (target4) at (0, -3.5) {$\targetmol$};
        \node[andnode, right = of target4] (r2) {$r_2$};
        \draw[->] (target4) -- (r2);
        \node[ornode, right = of r2] (mc) {$m_c$};
        \draw[->] (r2) -- (mc);
        \node[ornode, above right = of mc, xshift=-0.2cm, yshift=-0.2cm] (mb) {$m_b$};
        \draw[->] (r2) |- (mb);
        \node[ornode, below right = of mc, , xshift=-0.2cm, yshift=+0.2cm] (md) {$m_d$};
        \draw[->] (r2) |- (md);

    \end{tikzpicture}

    b) $\enumplans_{\targetmol}\left(\gG'\right)$

\end{minipage}

    \caption[AND/OR graph and its synthesis plans.]{
       \textbf{a)} graph $\gG'$ with (backward) reactions
            $\targetmol\Rightarrow m_a+m_b\,(r_1)$,
            $\targetmol\Rightarrow m_b+m_c+m_d\,(r_2)$,
            and $m_a\Rightarrow m_e\,(r_3)$.
            Dashed box illustrates expansion of $m_c$. %
        \textbf{b)} All synthesis plans in $\enumplans_{\targetmol}(\gG')$.
    }
    \label{fig:retrosynthesis-background-schematic}
\end{figure}

Various search algorithms have been proposed to find synthesis plans in $\gG$.
Although they use different mechanisms,
at a high level they all work similarly.
First, they initialize the explicit search graph $\gG'\subseteq\gG$
by $\gG'\gets\{\targetmol\}$ (i.e.\@ just the target molecule).
Nodes whose children have not been added to $\gG'$
form the \emph{frontier} $\frontier(\gG')$.
Then, at each iteration $i$ they select a frontier molecule $m_{(i)}\in\frontier(\gG')$
(necessarily $\targetmol$ on the first iteration),
query $B$ to find reactions which produce $m_{(i)}$,
then add these reactions and their corresponding reactant molecules to the explicit graph $\gG'$.
This process is called \emph{expansion},
and is illustrated for $m_c$ in Figure~\ref{fig:retrosynthesis-background-schematic}a.
Search continues until a suitable synthesis plan is found or until the computational budget is exhausted.
Afterwards,
synthesis plans can be enumerated from $\gG'$.
This general procedure is described in Algorithm~\ref{alg:general-retrosynthesis-search}.

\begin{algorithm}[th]
\caption{General retrosynthesis search algorithm.}\label{alg:general-retrosynthesis-search}
\begin{algorithmic}[1]
\REQUIRE target molecule $\targetmol$, backward reaction model $B$
\STATE $\gG'\gets \{\targetmol\}$
\FOR{$i$ in $1,2,\ldots$}
    \STATE Compute $\frontier(\gG')$\hfill\COMMENT{Set of frontier nodes}
    \STATE\label{general retrosynthesis alg: node selection}
        Select $m_{(i)}\in\frontier(\gG')$ to expand\hfill\COMMENT{This is the core part of the search algorithm}
    \STATE $R\gets B(m_{(i)})$\hfill\COMMENT{Use backwards reaction model to find reactions with product $m_{(i)}$}
    \FOR{$r\in R$}
        \STATE Add $r$ and its reactant molecules to $\gG'$
    \ENDFOR
    \IF{computational budget is exhausted}
        \RETURN $\gG'$\hfill\COMMENT{Terminate search}
    \ENDIF
\ENDFOR
\end{algorithmic}
\end{algorithm}

The key difference between algorithms is in line~\ref{general retrosynthesis alg: node selection}:
which frontier node to select for expansion.
The most popular retrosynthesis search algorithms compute some sort of metric
of synthesis plan quality, and use a 
\emph{search heuristic} to guide the search towards high-quality synthesis plans.
The remainder of this section will introduce these algorithms in more detail.


\subsection{Retro* search algorithm}
\label{sec:background:retro*}

Retro* is a greedy best-first search algorithm to find minimum-cost synthesis plans
on AND/OR \emph{tree}-structured search graphs
\citep{chen2020retro}.
It is functionally identical to the previously proposed AO* search algorithm
\citep{chang1971admissible,martelli1978optimizing,nilsson1982principles,mahanti1985andor},
although \citet{chen2020retro} do not mention this connection (most likely because they were not aware of it).
In retro*, the cost of a synthesis plan is defined as:
\begin{equation}
    c_{\text{plan}}(T) = \sum_{
        \underbrace{r\in T\cap \rxnspace}_{\text{reactions in }T}
        } c_{\text{rxn}}(r) 
    + \sum_{
        \underbrace{m\in \frontier(T)}_{\text{starting molecules in }T}
        } c_{\text{mol}}(m)
    \ ,
\end{equation}
namely the sum of execution costs
for each reaction
(given by $c_{\text{rxn}}:\rxnspace\mapsto[0, \infty]$)
and purchase costs for each starting molecule
(given by $c_{\text{mol}}:\molspace\mapsto[0,\infty]$).\footnote{
    Note that this formalism explicitly allows for infinite costs.
}
This essentially assumes that a high-quality synthesis plan will be composed
of individually low-cost reactions
and low-cost molecules (where ``cost'' does not necessarily mean monetary cost).

Let $c^*:\molspace\cup\rxnspace\mapsto[0,\infty]$ be a function which yields
the minimum cost synthesis plan\footnote{For any molecule, not just the target molecule $\targetmol$}
which contains the input molecule or reaction.
By virtue of using an AND/OR \emph{tree} with separate costs for each reaction and molecule,
such a minimum cost synthesis plan will contain sub-plans (i.e.\@ sub-trees)
which also have minimum costs.
This means $c^*$ will satisfy the recursive equations:
\begin{align}
    c^*(m) &= \min\left[
        c_{\text{mol}}(m),
        \min_{r\in \children_{\gG'}(m)} c^*(r)
    \right] \label{eqn:retro* min cost mol} \\
    c^*(r) &= c_{\text{rxn}}(r) + \sum_{m\in\children_{\gG'}(r)} c^*(m) \label{eqn:retro* min cost rxn} \ ,
\end{align}
where $\children_{\gG'}(n)$ denotes the set of children of node $n$ in the search graph $\gG'$.
Furthermore, as trees are acyclic, these recursive equations completely define $c^*$
(using the convention that the minimum of an empty set is $+\infty$).

As a greedy algorithm,
retro* selects frontier nodes for expansion if their expansion is
predicted to lead to a low-cost synthesis plan
for the target molecule $\targetmol$.
A search heuristic $h:\molspace\mapsto[0,\infty]$
is used to predict the future $c^*$ value of frontier molecules should they be expanded.
This leads to an estimated minimum cost function 
$\hat c^*:\molspace\cup\rxnspace\mapsto[0,\infty]$ defined recursively as:
\begin{align}
    \hat c^*(m) &= \begin{cases}
        \min\left[ c_{\text{mol}}(m), h(m)\right] & \text{if } m\in\frontier(\gG') \\
        \min\left[
            c_{\text{mol}}(m), 
            \min_{r\in \children_{\gG'}(m)} \hat c^*(r)
        \right] & \text{if } m\notin\frontier(\gG')
    \end{cases} \label{eqn:retro* min cost mol with heuristic}\\
    \hat c^*(r) &= c_{\text{rxn}}(r) + \sum_{m\in\children_{\gG'}(r)} \hat c^*(m) \ .%
    \label{eqn:retro* min cost rxn with heuristic}
\end{align}
The key difference between these equations and
equations~\ref{eqn:retro* min cost mol}--\ref{eqn:retro* min cost rxn} for $c^*$
is that frontier molecules (and only frontier molecules) can use
the heuristic $h$ as their cost instead of $c_{\text{mol}}$.

Once the estimated minimum cost function $\hat c^*$ is computed for all nodes in the search graph $\gG'$,
$\hat c^*(\targetmol)$ gives the cost of the best (predicted) synthesis plan for the target molecule $\targetmol$.
Frontier nodes on this synthesis plan can be found by descending the tree from $\targetmol$
and selecting the child with the lowest $\hat c^*$ value at each step.
This whole procedure is used to implement step~\ref{general retrosynthesis alg: node selection}
of algorithm~\ref{alg:general-retrosynthesis-search}.

\subsection{Monte Carlo Tree Search}

Monte Carlo Tree Search (MCTS) is a family of algorithms from reinforcement learning,
typically used to estimate the maximum reward action in a game tree
\citep{browne2012survey}.
It was first applied to retrosynthesis by \citet{segler2018planning},
who used an OR tree search graph (illustrated in Figure~\ref{fig:or-graph-schematic})
to allow retrosynthesis to be formulated as a Markov decision process (MDP).
In this kind of search tree, synthesis plans correspond to paths in the tree.

First, MCTS initializes a ``visit count'' for each node in the search graph $\gG'$.
At each step of MCTS, starting at the root node of the search graph
(with molecule set $\{\targetmol\}$),
it descends the tree choosing nodes which maximise
\begin{equation}
    \frac{W(s_t,a)}{N(s_t,a)} + cP(s_t,a)\frac{\sqrt{N(s_{t-1},a_{t-1})}}{1+N(s_t,a)}\ ,
\end{equation}
where $W(s_t,a)$ is the total reward accumulated while taking action $a$ to reach state
$s_t$, $N(s_t,a)$ is the number of times when the algorithm has performed action $a$
to reach state $s_t$ and $P(s_t,a)$ is some sort of prior probability
of performing action $a$ to reach $s_t$,
and $c$ is a constant.
At the end of this descent, a frontier node will always be reach reached
(because an OR tree is acyclic by construction).
If this frontier node has not been visited before,
a reward is computed with the search heuristic.
Otherwise, the node is expanded and one of its children is visited.
Terminal nodes (i.e.\@ nodes which cannot be expanded) are given a reward from the reward
function.
After the visit, the reward is propagated back up the tree towards the root node,
and the visit count of all nodes in the path is incremented by 1.
This whole procedure is used to implement step~\ref{general retrosynthesis alg: node selection}
of algorithm~\ref{alg:general-retrosynthesis-search}.
Further details of MCTS can be found in \citet{segler2018planning}
or in the paper accompanying the AiZynthFinder package \citep{genheden2020aizynthfinder}.

\subsection{Breadth-first search}

Breath-first search 
simply keeps a first-in-first-out queue of frontier nodes,
and expands them in order.
Unlike retro* or MCTS, it does not use a search heuristic to guide the search.
For this reason, it can be thought of as a ``naive'' or ``uninformed'' algorithm,
and is mainly useful as a simple baseline.
