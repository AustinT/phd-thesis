\section{Probabilistic models for molecular properties}
\label{sec:background:models}

The main contributions presented in this thesis are algorithms which \emph{use}
models to perform supervised learning (\S\ref{sec:background:supervised learning})
or molecule optimisation (\S\ref{sec:background:molopt}),
while being mostly agnostic to the details of the models themselves.
Furthermore, since molecules can be represented as sequences and vectors (\S\ref{sec:background:mols}),
and since these data types are arguably the most studied data types in all of machine learning,
there exists an enormous number of machine learning models which \emph{could} be applied to molecules.
Therefore, this section will not even attempt to present a comprehensive introduction to such methods;
if interested a reader may consult any number of high-quality textbooks or reviews
\citep{bishop2006pattern,Goodfellow-et-al-2016deep-learning,murphy2022probabilistic}.
Instead, this section will present and contextualize a small number of model classes which appear in this thesis,
and will also explain their probabilistic interpretation (which may not be familiar to all readers).

\subsection{Probabilistic linear models}
\label{sec:background:linear models}

Given a \emph{feature} function $\phi:\mathcal X \mapsto \R^M$,
linear regression defines $\hat f: \mathcal X \mapsto \R$ as
\begin{equation}
    \hat f (x) = \bm{w} \cdot \phi (x) + b \quad .
\end{equation}
This model has a vector of \emph{weight} parameters $\bm{w}\in\R^M$
and a \emph{bias} parameter $b\in\R$.
In a non-probabilistic regression setting $\hat f$ is directly used to predict
labels $y$ (typically called ``linear regression'').
In the probabilistic setting however,
$\hat f$ is connected to $y$ via an \emph{observation model}
which assigns a probability to every possible outcome for $y$.

For regression, this observation model is most often
a Gaussian distribution centred on $\hat f(x)$,
with probability density function
\begin{equation}\label{eqn:background:gaussian noise}
    p_{\bm{w}, b, \sigma^2}(y|x) = \mathcal{N}(y; \hat f(x), \sigma^2)\quad .
\end{equation}
Here, $\mathcal N(y; a, b)$ is the probability density
of a Gaussian distribution with mean $a$
and variance (or covariance matrix) $b$ at $y$.
The parameter $\sigma^2$ is also called the \emph{noise} parameter.

For \emph{binary} classification ($C=2$),
the model
\begin{equation}
    \probability_{\bm{w}, b}\left(y=1|x\right) = \frac{e^{\hat f(x)}}{1 + e^{\hat f (x)}}
\end{equation}
is typically used ($\probability$ denotes the probability of an event).\footnote{
    This is also called \emph{logistic regression}.
}
For $C>2$ a single linear model is insufficient to assign probabilities to all outcomes;
instead a set of linear models with parameters $(\bm{w_1}, b_1),\ldots,(\bm{w}_C, b_C)$
is created, with
\begin{equation}
    \probability_{\bm{w}_1,b_1,\ldots,\bm{w}_C,b_C} \left( y = i | x \right) \propto e^{\hat f_i(x)}\quad .
\end{equation}

Linear models are not used directly in this thesis, but
form the basis for more complex models introduced in the following sections.

\subsection{Gaussian processes and Bayesian linear models}
\label{sec:background:gps}

The crux of Bayesian modelling is to treat the model itself as a random variable,
typically by introducing a \emph{prior} distribution on the model parameters.
If a standard normal $\mathcal N(\bm{0}, \mI)$ prior is put on the linear model weights
$\bm{w}$, this induces the distribution
\begin{equation}
    \begin{bmatrix}
        \hat f (x_1) \\
        \vdots \\
        \hat f (x_N)
    \end{bmatrix}
    \sim \mathcal N \left(b \bm{1}, \mK \right)
\end{equation}
over the outcomes $\hat f (x_1), \ldots, \hat f (x_N)$
(which are random variables by implication of $\bm{w}$ being a random variable).
Here, $\bm{1}$ denotes a vector of $1$s,
and $\mK$ is a matrix whose entries correspond to inner products between feature vectors
\begin{equation}\label{eqn:background:kernel function defn}
    \emK_{i,j}=k(x_i,x_j):=\phi(x_i) \cdot \phi(x_j) \quad .
\end{equation}

The function $k(\cdot, \cdot)$ is called the \emph{kernel function}
(discussed further in \S\ref{sec:background:mol-kernels}).
Because the distribution of $\hat f$ depends only on features $\phi$
via the inner product function $k$,
it is common to define Bayesian linear models directly using $k$.
In this context, such models are usually referred to as \emph{Gaussian processes} (GPs)
and can be thought of directly defining a distribution over functions $\mathcal X\mapsto \R$:
\begin{equation}
    \hat f \sim \mathcal{GP}(\mu(\cdot), k(\cdot, \cdot)) \quad .
\end{equation}

Measurements $y_1,\ldots,y_N$ can then be described as sampled from distributions
induced by random samples of $\hat f$.
However, what really makes GPs useful for machine learning is that
if the observation model is also Gaussian (equation~\ref{eqn:background:gaussian noise}),
then the Bayesian posterior distribution over $\hat f$ is \emph{also a Gaussian process}
(although with a different mean and kernel function which depend on the observed data).
With a constant noise value of $\sigma^2$, the posterior distribution over query points
$x^*_1, \ldots, x^*_Q$ 
given observations
$((x_1, y_1), \ldots (x_N, y_N))$
has the form
\begin{equation}
    \begin{bmatrix}
        \hat f(x^*_1) \\
        \vdots \\
        \hat f(x^*_Q)
    \end{bmatrix}
    \sim \mathcal{GP} \left(\tilde \mu (\cdot), \tilde k(\cdot, \cdot) \right)
\end{equation}
where
\begin{align}
    \tilde \mu(x^*) &= \mu(x^*) 
    + \begin{bmatrix}
        k(x^*, x_1) \\
        \vdots \\
        k(x^*, x_N)
    \end{bmatrix}^T
    \left(\mK(X, X) + \sigma^2\mI\right)^{-1}
    \begin{bmatrix}
        y_1 - \mu(x_1) \\
        \vdots \\
        y_N - \mu(x_N) \\
    \end{bmatrix}\quad,  \\
    \tilde k(x^*_i, x^*_j) &=
    k(x^*_i, x^*_j) -
    \begin{bmatrix}
        k(x^*_i, x_1) \\
        \vdots \\
        k(x^*_i, x_N)
    \end{bmatrix}^T
    \left(\mK(X, X) + \sigma^2\mI\right)^{-1}
    \begin{bmatrix}
        k(x_1, x^*_j) \\
        \vdots \\
        k(x_N, x^*_j)
    \end{bmatrix}\quad,  \\
    \mK(X,X)_{i,j} &= k(x_i,x_j)
\end{align}
\citep{rasmussen2006gp}.
This also allows the marginal likelihood of the data to be computed in closed form.
For a GP with a constant noise value $\sigma^2$,
mean function $\mu(\cdot)$, and kernel function $k(\cdot, \cdot)$,
the (log) marginal likelihood of the data $X=\left[x_1,\ldots,x_N\right]$
is given by
\begin{equation}\label{eqn:background:gp-marginal-likelihood}
    \log{p(\bm{y}|X)} =
        -\frac{1}{2}\bm{y}^T
            \left(\mK(X, X) + \sigma^2\mI\right)^{-1}
            \bm{y}
        -\frac{1}{2}\log\left|\mK(X, X) + \sigma^2\mI\right|
        -\frac{N}{2}\log(2\pi)
\end{equation}
\citep{rasmussen2006gp}.
GP models are often selected to maximise the marginal likelihood of the data
by maximising equation~\ref{eqn:background:gp-marginal-likelihood}
with respect to the parameters of the mean and kernel functions.
    
\subsection{Gaussian process kernels \emph{for molecules}}
\label{sec:background:mol-kernels}

Equation~\ref{eqn:background:kernel function defn}
defined the kernel function $k(\cdot, \cdot)$ as the inner product of feature vectors.
However, in practice it is much more common to define the kernel function directly.
Any kernel function which is \emph{positive definite}
(i.e.\@ no kernel matrix formed from the kernel function will have negative eigenvalues)
will implicitly correspond to an inner product between feature vectors
in some Hilbert space, which may be infinite-dimensional
\citep{scholkopf2002learning}.
Since the kernel provides the primary inductive bias of GP models,
it is important to choose a kernel function carefully.

For vector-valued data in $\R^n$,
the most common kernels are the \emph{radial basis function} (RBF) kernel
\begin{equation}\label{eqn:rfb kernel defn}
    k(x, x') = \exp\left(-\frac{\|x - x'\|^2}{2\ell^2}\right)
\end{equation}
or the \emph{Mat\'ern} family of kernels, indexed by a parameter $\nu$.
A common value for $\nu$ is $\nu=5/2$,
which yields the kernel function
\begin{equation}\label{eqn:matern52 kernel defn}
    k(x, x') = \left(1 + \frac{\sqrt{5}\|x - x'\|}{\ell} + \frac{5\|x - x'\|^2}{3\ell^2}\right)
    \exp\left(-\frac{\sqrt{5}\|x - x'\|}{\ell}\right) \quad .
\end{equation}
Both of these kernels encode the assumption that input vectors
with a small Euclidean distance are likely to have similar outputs.

Vector valued kernels can be applied to some molecular representations
(e.g.\@ embeddings) but are not directly applicable to molecular graphs.
For fingerprint representations of molecules it is useful to define set-valued kernels.
A notable set kernel which will be used in several places throughout this thesis is the
\emph{Tanimoto kernel},
also called the \emph{Jaccard kernel}
\citep{jaccard1912distribution,tanimoto1958elementary}.
This kernel has been widely used in cheminformatics to compare molecular fingerprints
\citep{ralaivola2005graph,bajusz2015tanimoto,o2016comparing,miranda2021differential}.
It can be defined in several equivalent ways.
For plain sets, the definition
\begin{equation}\label{eqn:background:tanimoto-for-sets}
    T(S_1, S_2) = \frac{\left| S_1 \cap S_2 \right|}{\left| S_1 \cup S_2 \right|}
\end{equation}
suffices.
This definition can easily be extended to multi-sets.
Letting $m_i(x)$ denote the multiplicity of $x$ in multi-set $S_i$,
we define
\begin{equation}\label{eqn:background:tanimoto-for-multisets}
    T(S_1, S_2) = \frac{\sum_{x\in S_1\cup S_2}\min\left(m_1(x), m_2(x)\right)}{\sum_{x\in S_1\cup S_2}\max\left(m_1(x), m_2(x)\right)} \hfill .
\end{equation}
Note that equations~\ref{eqn:background:tanimoto-for-sets} and \ref{eqn:background:tanimoto-for-multisets}
are consistent and compatible using standard notions of intersection/union for multi-sets.
This kernel encodes the inductive bias that input sets with a high degree of overlap
are likely to have similar outputs.
When applied to molecular fingerprints, this is equivalent to assuming
that molecules with similar substructures are likely to have similar properties.

The Tanimoto kernel has the useful property that $1-T(\cdot, \cdot)$ is a valid distance metric,
typically called the \emph{Jaccard/Soergel distance}
\citep{marczewski1958certain,levandowsky1971distance}.
This makes it useful in nearest-neighbour methods in addition to GP models
(although this will not be exploited in this thesis).

Combining the Tanimoto kernel with a molecular fingerprinting function
creates a kernel directly over graph space.
Prior works have also developed many other kernels 
which operate directly on graph-structured inputs 
\citep{nikolentzos2021graph}.
These kernels generally also perform some sort of comparison of substructures present within the two input graphs.
For example,
\citet{moss2020boss} converts graphs into SMILES strings and defines a kernel based on substring matching.
\citet{korovina2020chembo} proposes a kernel based on graph optimal transport.
These kernels are not used directly in this thesis and so will not be discussed further.


\subsection{Fast approximations for Gaussian processes}
\label{sec:background:approx-gp-inference}

Regardless of the kernel used, GP models are notorious for scaling poorly to large datasets.
Evaluating the posterior covariance requires computing the kernel value between all
pairs of input data in $x_1,\ldots,x_N$ (time complexity $\mathcal O (N^2)$)
and inverting a regularized version of this matrix (time complexity $\mathcal O (N^3)$).
Even if $N$ is small, producing a full covariance matrix between $Q$
query points still has $\mathcal O(Q^2)$ cost,
and many operations such as sampling require inverting this matrix ($\mathcal O(Q^3)$).
For this reason approximations to GPs have been widely studied
\citep{liu2020scalable-gps}.

There are many different approaches which all aim to avoid expensive matrix computations in different ways.
One popular class of approaches are \emph{inducing point methods}.
These approximate a GP over $x_1,\ldots,x_N$
with a GP over a smaller set of $M$ pseudo-data points $z_1,\ldots,z_M\in\mathcal X$.
If $M\ll N$ these approximations will be significantly faster than the exact GP,
generally reducing $\mathcal O (N^2)$ operations to $\mathcal O(MN)$ and $\mathcal O (N^2)$ to $O(NM^2)$.
The inducing points function as a set of tunable parameters
and are generally chosen to ``cover'' the dataset.
A commonly-used model of this type is the stochastic
variational Gaussian process (SVGP) 
\citep{titsias2009variational,hensman2013gaussian}.

Another approach is \emph{random features} \citep{rahimi2007random}.
Given a kernel $k$, a \emph{random features map} 
is a random function
$f:\mathcal{X}\mapsto\mathbb{R}^M$, with the property that  $f(x)\cdot f(x')$ approximates $k(x,x')$ for every pair $x,x'\in\mathcal X$.
The approximation is often exact in expectation:
\begin{equation}\label{eqn:random-feature-definition}
    \mathbb{E}_f\left[f(x)\cdot f(x')\right]=k(x,x') \quad \text{for all }x,x'\in\mathcal X.
\end{equation}
Random features effectively create a rank-$M$ approximation to the kernel matrix $\mK(\mX, \mY) \approx f(\mX)^T f(\mY)$.
This results in the same reductions to computational cost as the inducing point approximations,
i.e.\@ at most linear in $N$.
A random features map is sometimes called a \emph{data-oblivious sketch}, to distinguish it from other \emph{data-dependent}
low-rank approximation methods which depend on a given dataset.
Examples of data-dependent low rank sketches are the Nystr\"{o}m approximation and leverage-score sampling \citep{drineas2005nystrom,drineas2012fast}.
Although data-dependent methods may result in lower approximation errors for a given dataset,
data-oblivious sketches are naturally parallelizable and useful in cases where the dataset  changes over time (e.g.\@ streaming or optimisation)
or for ultra-large datasets which may not fit in memory.

Other approaches accept the $\mathcal O (N^2)$ construction of the kernel matrix,
but try to approximate its inversion in less than $\mathcal O (N^3)$ complexity.
Approaches in this category include conjugate gradient methods
\citep{gardner2018gpytorch,wang2019exact}
and stochastic gradient methods
\citep{antoran2023samplingbased,lin2023sampling-sgd,lin2024stochastic}.
However, these are not used in this thesis.

Importantly, note that these methods are ultimately \emph{approximations}
to exact Gaussian processes and therefore will in general provide different predictions.
Approximate GPs are best thought of as sacrificing prediction quality
for computational tractability.

\subsection{Deep Learning and Neural Networks}
\label{sec:background:deep learning for molecules}

Deep learning describes an enormous family of models which map inputs to outputs
using a sequence of non-linear transformations (typically differentiable).
Deep learning has proliferated enormously over the past decade,
producing a deluge of model types.
A thorough introduction to deep learning can be found elsewhere
\citep{Goodfellow-et-al-2016deep-learning,murphy2022probabilistic}.
Here we will only briefly mention the neural network types which are used in this thesis.

\begin{itemize}
    \item \textbf{Feedforward neural networks},
        also called a multi-layer perceptron (MLP) for historical reasons,
        transform vector inputs to vector outputs using an alternating sequence of matrix multiplications
        and non-linear transformations referred to as ``activation functions''.
        They are described in detail in \citet[chapter 6]{Goodfellow-et-al-2016deep-learning}.
    \item \textbf{Graph neural networks} (GNNs)
        accept graphs as inputs and produce vectors as outputs.
        Many GNNs effectively pass ``messages'' between nodes,
        which are weighted by learnable parameters to produce an embedding for each node.
        These embeddings are aggregated to produce an overall graph embedding.
        A more thorough introduction to GNNs is given in
        \citet{zhou2020graph}.
\end{itemize}

Throughout this thesis neural networks will simply be treated as parameterized
functions $f_{\bm{\theta}}:\molspace \mapsto \R^n$
whose outputs are differentiable with respect to the parameters $\bm{\theta}$.
The outputs of $f_{\bm{\theta}}$ can be used as the input features to a linear model
(\S\ref{sec:background:linear models}),
and thereby can form probabilistic models in the same way.

