
\chapter{Introduction}
\label{chapter:intro}

\ifpdf
    \graphicspath{{chapter01-intro/figures}}
\else
    \graphicspath{}
\fi

Despite limited time, resources, and ingenuity, humanity has discovered molecules
whose properties give us extraordinary control over its environment.
Penicillin transformed bacterial infections from a deadly disease into an inconvenience.
Pesticides like DDT have eradicated malaria from entire continents.
Birth control pills have given women unprecedented control over their own reproduction.
Optimistically, still yet-to-be discovered molecules will cure cancer,
enable a worldwide transition away from fossil fuels,
or slow human ageing.

Hopefully these examples make a convincing case
that molecule discovery is a worthwhile endeavour.
Historically, many important molecules have been discovered
\emph{serendipitously}---
essentially by accident.\footnote{
    A great book on serendipitous discoveries in medicine is
    \emph{Happy Accidents: Serendipity in Modern Medical Breakthroughs}
    by Morton Meyers.
    He describes serendipity as not simply ``luck'',
    but ``unexpected observation seized upon and turned to advantage by the prepared mind.''
    In other words, partially luck, but partially preparedness.
}
Although such serendipitous discoveries will doubtlessly continue in the future,
methods to discover molecules \emph{intentionally} and \emph{systematically} are clearly desirable.
We now consider how this may be achieved, starting with some definitions.

\section{Definition and challenges of ``molecule discovery''}

\subsection{What are molecules?}
\label{sec:what are molecules}

Although this question may seem overly pedantic,
there are many useful substances like metal alloys which
are not normally considered to be ``molecules''.
A definition of molecules is therefore helpful to understand the scope of this thesis.
Zumdahl's classic chemistry textbook \citep[page 52]{zumdahl2006chemistry}
provides the following explanation of a molecule:

\begin{quotation}
    The forces that hold atoms together in compounds are called \textbf{chemical bonds}.
    One way that atoms can form bonds is by \textit{sharing electrons}.
    These bonds are called \textbf{covalent bonds}, and the resulting collection of atoms is called a \textbf{molecule}.
\end{quotation}

Therefore, in this thesis we will simply consider to be a molecule
\emph{to be a collection of atoms connected by bonds}.\footnote{
    Of course, bonds themselves are also an abstraction (as anyone who has seen my ``bonds do not exist'' T-shirt will know).
    \citet[page 348]{zumdahl2006chemistry} provide some additional thoughts on this:
    \begin{quotation}
        It is important to note that the bond concept is a human invention.
        Bonds provide a method for dividing up the energy evolved when a stable molecule is formed from its component atoms.
        Thus in this context a bond represents a quantity of energy obtained from the overall molecular energy of stabilization
        in a rather arbitrary way. This is not to say that the concept of individual bonds is a bad idea.
        In fact, the modern concept of the chemical bond, conceived by the American chemists G.\@ N.\@ Lewis
        and Linus Pauling, is one of the most useful ideas chemists have ever developed.
    \end{quotation}
}
This definition is of course imperfect,
but undeniably includes most small molecule drugs and peptides,
which are the main envisioned application of the algorithms in this thesis.
Ultimately, this definition motivates representing molecules as mathematical graphs (\S\ref{sec:background:mols}).

\subsection{What is ``discovery''?}
\label{sec:intro:what is discovery}

Again, this question may seem pedantic,
but exactly what constitutes ``discovery'' is less clear than it seems.
For example, there are some molecules whose properties were known and exploited by ancient civilizations,
but were only chemically identified in the past few centuries (e.g.\@ many plant-based dyes).
Another situation worth considering is if a certain molecule is ``predicted'' to exist or to have certain properties,
but is not synthesized until much later.
In both of these cases, what should be considered the ``discovery'' is not clear.

In this thesis we will adopt the pragmatic definition that discovery is
\emph{both} the identification of a molecule \emph{and} the measurement of some property of interest.
In particular, because this definition requires both the molecule and a property to be known,
it makes sense to describe the relationship between the two as a mathematical function,
allowing molecule discovery to later be formalized as optimisation
(\S\ref{sec:background:problems}).

\subsection{What makes molecule discovery challenging?}

Discovering molecules is essentially a ``needle in a haystack'' problem.
There are many potential molecules to test (some estimates suggest $10^{60}$ \citep{bohacek1996art}),
but typically only a miniscule fraction of these molecules will possess any particular
property of interest (e.g.\@ curing bacterial infections).
This challenge is furthered by the fact that properties of interest
(e.g.\@ therapeutic outcomes)
are challenging to predict accurately and expensive to measure experimentally.
This limits the amount of information that can be acquired during the discovery process.

These factors make exhaustively testing every molecule infeasible
and the simple strategy of random testing ineffective.
To discover molecules effectively,
one must clearly have some knowledge about which molecules are likely to have the desired properties,
then choose which molecules to test very carefully in order to not ``waste'' any experiments.
Even with a lot of expert knowledge and large research budgets,
molecule discovery campaigns still fail routinely
(the statistic ``over 90\% of drug discovery projects fail'' is often cited,
e.g.\@ by \citet{hay2014clinical}).

\section{Why consider \emph{probabilistic machine learning}?}

Although computers may not know any more about chemistry than expert humans,
they are better at processing information, which is important as chemistry enters the ``big data'' era.
Modern high-throughput screening approaches can quickly produce thousands of measurements,
and ``virtual libraries'' can contain up to $10^{20}$ molecules \citep{hoffmann2019next}:
numbers far too large for even the smartest human to fully process and analyse.
Just being able to comprehensively consider all the data and all options for subsequent experiments
makes computer algorithms an important tool in molecule discovery.

Historically, the most popular class of algorithms used in molecular discovery has been physics-based models,
including docking \citep{pinzi2019molecular}, molecular dynamics \citep{durrant2011molecular},
and density-functional theory \citep{jones2015density}.
Such approaches are highly appealing to scientists as they are principled,
interpretable, and generalizable.
However, within these methods there is always a trade-off between accuracy and computational cost,
with the most accurate simulations being very expensive
(often taking days or weeks to run on large supercomputers).
Even given immense computational resources, the most accurate methods still make simplifying
assumptions and therefore will never perfectly match reality.
Ultimately, physics-based algorithms can be thought of as producing predictions ranging from
``cheap but very inaccurate'' to ``expensive but still imperfect.''
Chemists, understandably, find this trade-off frustrating and unsatisfactory.


With this backdrop, abandoning first-principles and simply ``curve fitting'' observed data emerges as a potentially appealing option.
We will broadly refer to all such techniques as \emph{machine learning}.
However, the algorithms in this thesis can all be categorized as \emph{probabilistic} machine learning algorithms:
this means they possess an interpretation as performing statistical inference over random variables,
for example the outcome of a measurement in a laboratory.
The use of randomness may seem odd to some readers, especially in the context of chemistry
where most quantities of interest are physicochemical properties
which are \emph{not} random.
However, randomness is a useful tool to model \emph{uncertainty}.
Since physicochemical properties are measured using imperfect equipment,
there is always uncertainty about quantities that one has measured,
and of course there is even more uncertainty about quantities that one has \emph{not} measured.
Estimates of uncertainty give ``confidence'' in the outcome of experiments before they are performed,
and are therefore incredibly useful for decision-making.

Aside from uncertainty, designing machine learning algorithms with a
probabilistic interpretation provides a principled framework for developing algorithms
rather than simply chaining together many steps in an \textit{ad-hoc} way.
This gives probabilistic machine learning algorithms an interpretable flair
which many scientists find appealing.

\section{Contributions of thesis}
\label{sec:intro:contributions}

This thesis describes several novel probabilistic machine learning algorithms 
applicable to molecule discovery.
There algorithms can predict properties of molecules,
generate molecules with desirable properties,
and produce synthesis plans for novel molecules.
One way to view these algorithms at a high level is as tools to aid in the
``design-make-test-analyse'' (DMTA) cycle of molecule discovery,
which is illustrated in Figure~\ref{fig:dmta-schematic}.
The steps in DMTA can be understood as follows:
\begin{enumerate}
    \item \textbf{Design}:
        choosing which molecule to test (e.g.\@ drawing the desired chemical structure).
    \item \textbf{Make}:
        physically creating the molecule (or just purchasing it).
        This step can be skipped if the testing is entirely computational.
    \item \textbf{Test}: make some sort of measurement of the molecule.
    \item \textbf{Analyse}: interpret the results of this measurement alongside previous measurements.
        Generally the goal is to make a decision based on the results;
        therefore this step is often tightly coupled with the ``design'' step.
\end{enumerate}

\begin{figure}[t]
    \begin{tikzpicture}[node distance=3cm,>={Stealth[scale=2]}]
    \node (design) [draw, rectangle, minimum width=3cm, minimum height=1cm] {Design};
    \node (make) [draw, rectangle, minimum width=3cm, minimum height=1cm, right=of design] {Make};
    \node (test) [draw, rectangle, minimum width=3cm, minimum height=1cm, below=of make] {Test};
    \node (analyse) [draw, rectangle, minimum width=3cm, minimum height=1cm, below=of design] {Analyse};
    
    \draw [->] (design) -- (make);
    \draw [->] (make) -- (test);
    \draw [->] (test) -- (analyse);
    \draw [->] (analyse) -- (design);
    
    \draw [decorate, decoration={brace, amplitude=10pt, raise=5pt}, yshift=0pt, xshift=0] 
    (analyse.south west) -- (design.north west) node [black, midway, xshift=-2cm, align=left] {
        Chapters~\ref{chapter:lso}, \ref{chapter:adkf}, \ref{chapter:trf}: \\
        Property \\ prediction \\ models \\
        (defined \S\ref{sec:background:models}) \\
    };

    \draw [decorate, decoration={brace, amplitude=10pt, raise=5pt}, rotate=90] 
    (design.north west) -- (make.north east) node [black, midway, xshift=0, yshift=1cm, align=left] {
        Chapter~\ref{chapter:rfb}: retrosynthesis \\
        (defined \S\ref{sec:background:synthesis}) \\
    };
\end{tikzpicture}

    \caption[Schematic of the design-make-test-analyse (DMTA) cycle]{
        Schematic of the design-make-test-analyse (DMTA) cycle for molecule discovery
        and how the algorithms presented in this thesis fit into it.
    } 
    \label{fig:dmta-schematic}
\end{figure}

Each content chapter of this thesis describes a specific algorithm which has been
published as a conference paper in a highly-regarded machine learning conference:
\begin{itemize}
    \item Chapter~\ref{chapter:lso}
        describes a method to train a generative model on weighted data
        to produce molecules with desirable properties.
        Unlike previous methods in this class,
        this method is able to ``zoom in'' on promising regions of molecule space.
        Its contents formed the paper
        \emph{Sample-Efficient optimization in the Latent Space of Deep Generative Models via Weighted Retraining},
        published in Neural Information Processing Systems (NeurIPS) 2020
        \citep{tripp2020sample}.
    \item Chapter~\ref{chapter:adkf}
        describes a method to fit Gaussian process models whose kernels contain neural networks with millions of parameters.
        Its noteworthy feature is to avoid overfitting and underfitting issues experienced by previous methods.
        Its contents formed the paper
        \emph{Meta-learning Adaptive Deep Kernel Gaussian Processes for Molecular Property Prediction},
        published in the International Conference on Learning Representations (ICLR) 2023
        \citep{chen2022meta}.
    \item Chapter~\ref{chapter:trf}
        describes an approximation to scale Tanimoto-kernel Gaussian process models to large datasets.
        Despite the Tanimoto kernel being a popular choice for molecular property prediction,
        no previous work had proposed a random feature approximation for it.
        Its contents formed the paper
        \emph{Tanimoto Random Features for Scalable Molecular Machine Learning},
        published in NeurIPS 2023
        \citep{tripp2023tanimoto}.
    \item Chapter~\ref{chapter:rfb}
        describes a novel probabilistic formulation of synthesizing molecules.
        Unlike previous algorithms, this method is able to handle uncertainty
        about reaction outcomes in a principled way.
        Its contents formed the paper
        \emph{Retro-fallback: retrosynthetic planning in an uncertain world},
        published in ICLR 2024 \citep{tripp2023retrofallback}.
\end{itemize}

The relationship between these algorithms and the DMTA cycle is illustrated in Figure~\ref{fig:dmta-schematic}.
The algorithms in chapters~\ref{chapter:lso}--\ref{chapter:trf} are used to propose molecules to test
based on the results of previous tests,
either by directly proposing molecules to test,
or simply providing probabilistic predictions predicting what the results of testing a molecule could be.
They can be viewed as jointly performing the ``analysis'' and ``design'' steps of DMTA.
In contrast, chapter~\ref{chapter:rfb} describes a novel algorithm to propose synthesis plans
for molecules which have been identified as desirable, using the formalism of ``retrosynthesis''.
It is primarily useful for the ``make'' step (since in practice this is often just executing a synthesis),
but also has implications for the ``design'' step as it can be used to propose molecules which are easier to synthesize.
All these algorithms can be incorporated into a ``Bayesian optimisation''
approach to DMTA, which will be described in \S\ref{sec:background:bayesopt}.

The work presented in this thesis is not all the work done during my PhD.
I chose only to include works which propose new probabilistic machine learning algorithms
for molecule discovery, and where I was also either the first author or made significant contributions.
However, I also contributed to the following papers:
\begin{itemize}
    \item \href{https://github.com/dockstring/dockstring}{\textsc{DockString}},
        a software package and benchmark suite based on molecular docking,
        published in the Journal of Chemical Information and Modeling  %
        \citep{ortegon2021dockstring}.
    \item A reinforcement learning algorithm for retrosynthesis \citep{liu2023pdvn},
        published in ICML 2023.
    \item An efficient approximate inference method for Gaussian processes,
        published in ICLR 2024 \citep{lin2024stochastic}.
    \item The \href{https://github.com/leojklarner/gauche}{GAUCHE} software library for Gaussian processes over molecules \citep{griffiths2024gauche},
        published in NeurIPS 2023.
    \item The \href{https://github.com/microsoft/syntheseus/}{\texttt{syntheseus}} software library for retrosynthesis,
        currently on arXiv \citep{maziarz2023re}.
\end{itemize}

\section{Guide to reading this thesis}

``Algorithms for molecule discovery'' is an inherently interdisciplinary topic,
and I hope this thesis can be useful to a diverse audience of readers.
However, since the content is formed from machine learning conference papers,
it is primarily written with a machine learning audience in mind.
Each content chapter describes one conference paper,
and the contents of each chapter are largely identical to that of the corresponding paper
except for the following changes.

First, the standard 8--10 page limit of machine learning conference papers is not particularly conducive
to lengthy explanations of background materials, leading to very terse introductions of key concepts
like Gaussian process models.
I have moved much of this material into a dedicated background chapter (Chapter~\ref{chapter:background})
and elaborated upon the explanations.

Second, the introduction to each conference paper contains a few paragraphs motivating machine learning for molecules.
As this content is largely redundant given this chapter and Chapter~\ref{chapter:background},
I have therefore changed section~1 of each chapter from ``introduction'' to ``motivation''
and replaced the content with a more focused and nuanced motivation for each method.
Third, I added some content to the chapters which was originally relegated to the appendix of the corresponding conference papers.
Finally, since this thesis is written with the benefit of hindsight I have added a final ``retrospective''
section at the end of each chapter with additional analysis.

Importantly, despite these changes this thesis includes no new \emph{technical} content
beyond what is in the corresponding papers.
Readers who are already familiar with any of the conference papers from this thesis can
simply read section~1
of the corresponding chapter for additional motivation
then skip directly to the final section for some retrospective analysis.
