\begin{abstract}
    Discovering new molecules empowers humanity to solve problems in
    health, agriculture, energy, and more.
    The key challenge of molecule discovery is that the space of all possible molecules is vastly
    larger than the amount of molecules which we can test experimentally with our limited resources.
    Given this challenge, arguably the best option is to judiciously
    select which molecules to test based on both our current knowledge
    and our expectation of what information will be gained from each test.

    In machine learning, this approach is typically called \emph{Bayesian optimisation}
    and has been studied for many other problems,
    such as tuning hyperparameters of machine learning models.
    Although in principle Bayesian optimisation can be straightforwardly
    applied to the problem of discovering new molecules,
    the discrete nature of molecules means that new models and algorithms
    are needed to make Bayesian optimisation work in practice.

    This thesis presents various probabilistic machine learning algorithms
    which could be used within a Bayesian optimisation loop to discover new molecules.
    \emph{Latent space optimisation with weighted retraining} (chapter~\ref{chapter:lso})
    and \emph{adaptive deep kernel fitting with implicit function theorem} (chapter~\ref{chapter:adkf})
    are both algorithms which use a Gaussian process with a deep neural network kernel function
    to model the relationship between molecular structure and some property of interest.
    \emph{Tanimoto random features} (chapter~\ref{chapter:trf}) allows an established
    cheminformatics model to be applied (approximately) to large datasets.
    Finally, \emph{retro-fallback} (chapter~\ref{chapter:rfb})
    uses a novel probabilistic formulation
    of the retrosynthesis problem to estimate whether a molecule can be synthesized,
    and thereby determine whether it should be considered by Bayesian optimisation.
    Together, these algorithms form a suite of tools which could be used
    to discover new molecules automatically and intelligently.

\end{abstract}
